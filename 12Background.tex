\chapter{Background}
\label{chap:bg}
\section{Partial functions}
We require a notion of partial functions to model various operations that require lookup by a key.
\begin{definition}
\label{def:partial}
A partial function $f : A \mapsto B$ is only defined on a subset $A' \subseteq A$. For all $x \in A$, either $f(x) \in B$ or $f(x)$ is undefined.
Partial functions consist of 2-tuples $x \to y$ so that $f(x) = y$ and each $x$ occurs at most once. We define the following operations on partial functions:
\begin{itemize}
    \item The domain of definition of a partial function is given by $\Dom(f) = A'$
    \item The restriction $f|_S : A \cap S \mapsto B$ of $f : A \mapsto B$ to a set $S$ is obtained by setting all elements in $A\setminus S$ to undefined
    \item The intersection $f \cap g : A_1 \cap A_2 \mapsto B$  of two partial functions $f : A_1 \mapsto B$ and $g : A_2 \mapsto B$ is given by $f \cap g = f|_{\Dom(g)}$
    \item The union $f \cup g : A_1 \cup A_2 \mapsto B$ is well-defined iff $\Dom(f) \cap \Dom(g) = \varnothing$. Then \[ (f \cup g)(x) = \begin{cases} 
    f(x) & \text{if } x \in \Dom(f) \\
    g(x) & \text{if } x \in \Dom(g) \\
    \end{cases} \]
    \item The set-minus operation $f \setminus g : A_1 \mapsto B$ is given by $f|_{A \setminus \Dom(g)}$
    \item The symmetric difference $f \oplus g : A_1 \cup A_2 \mapsto B$ is given by $(f \setminus g) \cup (g \setminus f)$
\end{itemize}


\end{definition}
\section{Paths}
Representation of heap structure requires a way to model field and array access sequences. While in actual programs paths can be described by a set of finite sequences, static analysis requires a more permissive language that can model potentially infinite paths. First we introduce a few auxiliary definitions:
\begin{definition}
A \emph{heap edge} $\mathcal{H} = \mathcal{F} \cup \{ \text{arrayAccess} \}$ is either an instance field or a placeholder for an array index access, as in general we cannot statically determine the accessed array element.
\end{definition}
\begin{definition}
A \emph{heap sequence} $\mathcal{H}^*$ is a finite word over the alphabet containing all heap edges.
\end{definition}
\begin{definition}
\label{def:path}
An (abstract) \emph{heap path} $\mathcal{P}$ is a pattern over heap edges with the following syntax:
\grammarindent=120pt
\begin{grammar}
<path> ::= $\epsilon$
\alt <concretePrefix> <abstractSuffix>

<concretePrefix> ::= $\epsilon$
\alt <heapEdge>.<concretePrefix>

<heapEdge> ::= field | "arrayAccess"

<abstractSuffix> ::= $\epsilon$
\alt \{ <fieldEntries> \}, <abstractSuffix>

<fieldEntries> ::= <fieldEntry> 
    \alt <fieldEntry>, <fieldEntries> \\ so that each <heapEdge> occurs at most once in <fieldEntries>

<fieldEntry> ::= $<heapEdge>^{\left[n, <upperBound>\right]}$ where $n \in \mathbb{N}$

<upperBound> ::= 0 | 1 | *
\end{grammar}
\end{definition}

\begin{definition}[Semantics]
The semantics of the heap path syntax are defined by a set of inference rules and axioms. If a heap sequence $s \in \mathcal{H}^*$ matches a path $p \in \mathcal{P}$, then $s \Downarrow p$. Note that $h^n$ is the n-fold repetition of $h$, i.e. $h^n$ = $\underbrace{h \cdots h}_{n \text{ times}}$.
In the following inference rules,
\begin{itemize}
    \item $h$ denotes a heap edge
    \item $s$ and $\mathit{pre}$ denote a heap sequence
    \item $p$ denotes a heap path
    \item $\mathit{suf}$ denotes an abstract suffix
\end{itemize}
Let $\operatorname{adjustBounds}$ be a function that reduces the lower and upper bounds of a matching heap edge by the appropriate amount \[ \operatorname{adjustBounds}(hs, h^{[n,m]}, s) := hs \setminus \{ h^{[n,m]} \} \cup \{ h^{[\max(0, n - |s|_h), \max(0, m - |s|_h)]} \} \] where
for all $n : \mathbb{N}$, $* - n := *$ and $|s|_h$ denotes the amount of elements $h$ in a sequence $s$.
\infax[P-epsilon]{\epsilon \Downarrow \epsilon}
\infax[P-prefix-empty]{\mathit{pre} \Downarrow \mathit{pre}}
\infrule[P-prefix]{s_2 \Downarrow \mathit{suf}}{\mathit{pre}.s_2 \Downarrow \mathit{pre}\ \mathit{suf}}
\infax[P-entry-eps]{\epsilon \Downarrow h^{[0,n]} }
\infrule[P-entry-one]{n \le 1}{ h \Downarrow h^{[n,1]}}
\infrule[P-entry-many]{m \ge n}{h^m \Downarrow  h^{[n,*]} }
\infrule[P-entries-incomp]{\exists h^{[n,m]} \in \mathit{hs}, s_1 \Downarrow h^{[n,m]} \andalso s_2 \Downarrow \operatorname{adjustBounds}(\mathit{hs}, h^{[n,m]}, s_1), \mathit{suf}}{s_1.s_2 \Downarrow \mathit{hs}, \mathit{suf}}
\infrule[P-entries-order]{\exists h_1^{[n_1,m_1]} \in hs_1, s_1 \Downarrow h^{[n_1,m_1]} \andalso \forall hs_2 \in \mathit{suf}, \forall h_2^{[n_2,m_2]} \in hs_2, h_1 \ne h_2 \andalso s_2 \Downarrow \mathit{suf}}{s_1.s_2 \Downarrow hs_1, \mathit{suf}}
\end{definition}

\begin{definition}[Equivalence]
Two paths $p_1, p_2 : \mathcal{P}$ are equivalent (denoted $p_1 \equiv p_2$) iff for all heap sequences $s$, $s \Downarrow p_1$ iff $s \Downarrow p_2$.
\end{definition}

\begin{definition}[Concatenation]
The concatenation $p_1.p_2 : \mathcal{P}$ of two paths $p_1, p_2 : \mathcal{P}$ is an operation defined so that the following inference rule holds
\infrule[P-concat]{s_1 \Downarrow p_1 \andalso s_2 \Downarrow p_2}{s_1.s_2 \Downarrow p_1.p_2}
\end{definition}

\begin{definition}[Alternation]
The alternation $p_1|p_2 : \mathcal{P}$ of two paths $p_1, p_2 : \mathcal{P}$ is an operation defined so that the following inference rules hold
\infrule[P-alter-left]{s \Downarrow p_1}{s \Downarrow p_1|p_2}
\infrule[P-alter-right]{s \Downarrow p_2}{s \Downarrow p_1|p_2}
\end{definition}

\begin{remark}
The actual implementation of heap paths has the following structure:
\begin{enumerate}
    \item A concrete prefix sequence $pre \in \mathcal{H}^*$
    \item An abstract suffix consisting of a 4-tuple with
     \begin{enumerate}
         \item a flag whether the suffix matches the empty path $\epsilon$
         \item a partial function $oa : \mathcal{H} \mapsto \{0,1,*\}$  so that for every matching heap sequence, all heap edges $h$ contained in the heap sequence occur at most $oa(h)$ times
         \item a partial function $ua : \mathcal{H} \mapsto \mathbb{N}$ so that for every matching heap sequence, all heap edges $h$ occur at least $ua(h)$ times
         \item a partial order $f < g$ so that if $f < g$, then the sequence only matches if no $f$ occurs after $g$
     \end{enumerate}
\end{enumerate}
\end{remark}

\section{Heap-shape analysis}
A shape analysis is a static analysis that determines information about allocated data structures in a program. We require that the used shape analysis is sound, i.e. that the determined information is true for every program input.

\begin{definition}
A \emph{heap-shape analysis} is a path-sensitive analysis that provides a function $\operatorname{reachableAt} : (\mathit{Stmt}, \mathit{Refs}) \to \mathit{Refs} \mapsto \mathcal{P}$  which given a statement and a base reference provides a partial function of all local variables referring to an object reachable from the base reference to a \emph{heap path} describing where that object is reachable from the given base. \emph{Path-sensitive} means that the analysis doesn't simply check whether $x$ is reachable from $y$ but returns an overapproximation of all paths to $x$ from $y$.
\end{definition}
As long as the used heap-shape analysis is path-sensitive and sound, it is irrelevant what analysis is used exactly. In principle even non-path-sensitive shape analyses can be used by assuming a path matching every heap sequence for all reference.

\section{Soot}
Soot~\cite{vallee1999soot} is a Java framework which provides a set of intermediate representations and APIs for analysis and optimisation of Java bytecode. The primary intermediate language used for analyses is Jimple; other representations include BAF, a stack-based language, Shimple, a static single assignment variant of Jimple, and Grimple, a form of Jimple suitable for decompiling.

Soot's execution is divided into packs, with each pack containing phases---phases being where the actual analyses are run. Usually execution follows the following order:
\begin{enumerate}
    \item The Jimple body creation pack (\texttt{jb}) is applied to each method body, which transforms Java bytecode into Jimple code.
    \item The 4 following whole-program packs are applied. These contain all interprocedual analyses as each phase in this pack runs exactly once.
    \begin{enumerate}
        \item call graph pack (\texttt{cg})
        \item whole-jimple transformation pack (\texttt{wjtp})
        \item whole-jimple optimisation pack (\texttt{wjop})
        \item whole-jimple annotation pack (\texttt{wjap})
    \end{enumerate}
    \item After that, a set of per-method jimple transformation, optimisation and annotation packs is applied (\texttt{jtp}, \texttt{jop} and \texttt{jap})
    \item Finally, the Jimple bodies are converted into the bytecode precursor BAF in the \texttt{bb} pack and tags are aggregated in the \texttt{tag} pack, for example making sure that for each line number, only one tag exists.
\end{enumerate}

\subsection{Jimple}
\label{subsect:jimple}
Jimple~\cite{vallee1999soot} is a representation of Java bytecode which is well-suited for static analyses and program transformations. Firstly, Jimple is compact; it has essentially only 11 types of statements and 19 instructions in total, compared to more than 200 instructions in Java bytecode. Secondly, every statement is in 3-address form, i.e. instructions are kept as simple as possible, most of them having the form \texttt{x = y op z}. Additionally, the stack has been eliminated and replaced by local variables. All implicit references to stack positions are now explicit references to locals. Finally, all local variables are typed with either a primitive, class or interface type. Due to type erasure in Java bytecode, support for generic types is missing.

\subsubsection{Language description}
Jimple has 11 types of statements, out of which \emph{assignStmt}, \emph{identityStmt} and \emph{invokeStmt} are relevant to this thesis.
\begin{itemize}
    \item \emph{assignStmt} is either an assignment of an \emph{rvalue} to a local or an \emph{immediate} to a static field, an instance field or an array reference.
    An \emph{rvalue} is a static field or instance field access, an array reference, an \emph{immediate} or an expression and an \emph{immediate} is a local or a constant.
    \item \emph{identityStmt} is used to assign locals to special values such as \texttt{this}, parameter references or a caught exception.
    \item \emph{invokeStmt} invokes a method and ignores the result
\end{itemize}
An expression can be one of the following:
\begin{itemize}
    \item an usage of a binary operator such as \texttt{+} on two immediates
    \item a type cast
    \item a check whether an immediate is an instance of some type
    \item one of the following method invocations
    \begin{itemize}
        \item \texttt{specialinvoke}, which is, e.g., an invoke of a constructor method
        \item \texttt{interfaceinvoke}, which is a call of a method defined in an interface
        \item \texttt{virtualinvoke}, which is a regular method call
        \item \texttt{staticinvoke}, which calls a static method
    \end{itemize}
    \item an instantiation of a new reference type (note that this doesn't call the actual constructor)
    \item an instantiation of an array or multi-array of some type with dimensions stored in immediates.
    \item the length of an immediate in case the immediate is an array
    \item the negative of an immediate in case the immediate is a number
\end{itemize}

\begin{figure}
\grammarindent=110pt
\centering
\begin{grammar}
<stmt> ::= <assignStmt> | <identityStmt> | <gotoStmt>
    \alt <ifStmt> | <invokeStmt> | <switchStmt>
    \alt <monitorStmt> | <returnStmt> | <throwStmt> 
    \alt <breakpointStmt> | <nopStmt>

<assignStmt> ::= local "=" <rvalue>";"
    \alt field "=" <imm>";"
    \alt local.field "=" <imm>";"
    \alt local"["<imm>"]" "=" <imm>";"

<identityStmt> ::= local ":= @this:" type";"
    \alt local ":= @parameter"$n$":" type";"
    \alt local ":= @exception;"
    
<gotoStmt> ::= "goto" label";"

<ifStmt> ::= "if" <conditionExpr> "goto" label";"

<invokeStmt> ::= "invoke" <invokeExpr>";"

<switchStmt> ::= "lookupswitch" <imm> "{" \\
    "case" value$_1$": goto" label$_1$";" \\
    $\cdots$ \\
    "case" value$_n$": goto" label$_n$";" \\
    "default: goto "defaultLabel"; };"
    \alt "tableswitch" <imm> "{" \\
    "case" low": goto" lowLabel";" \\
    $\cdots$ \\
    "case" high": goto" highLabel";" \\
    "default: goto "defaultLabel"; };"

<monitorStmt> ::= "entermonitor" <imm>";" | "exitmonitor" <imm>";"

<returnStmt> ::= "return" <imm>";" | "return;"

<throwStmt> ::= "throw" <imm>";"

<breakpointStmt> ::= "breakpoint;"

<nopStmt> ::= "nop;"
\end{grammar}
\label{fig:grmJimpleStmt}
\caption{Jimple grammar---statements}
\end{figure}
\begin{figure}
\grammarindent=110pt
\centering
\begin{grammar}
<imm> ::= local | constant

<conditionExpr> ::= <imm> <condop> <imm>

<condop> ::= ">" | "<" | "=" | "!=" | "<=" | ">="

<rvalue> ::= <concreteRef> | <imm> | <expr>

<concreteRef> ::= field | local.field | local"["<imm>"]"

<invokeExpr> ::= "specialinvoke" local.method"("<imm>","$\cdots$","<imm>")"
    \alt "interfaceinvoke" local.method"("<imm>","$\cdots$","<imm>")"
    \alt "virtualinvoke" local.method"("<imm>","$\cdots$","<imm>")"
    \alt "staticinvoke" method"("<imm>","$\cdots$","<imm>")"
    
<expr> ::= <imm> <binop> <imm> 
    \alt "("type")" <imm>
    \alt <imm> "instanceof" type
    \alt <invokeExpr>
    \alt "new" refType
    \alt "newarray("type")["<imm>"]"
    \alt "newmultiarray("type")["<imm>"]" $\cdots$ "["<imm>"]" "[]*"
    \alt "length" <imm>
    \alt "neg" <imm>
    
<binop> ::= "+" | "-" | "*" | "/" | "\%" | "rem" | "<<"  | "<<<"
    \alt  ">>" | "&" | "|" | "cmp" | "cmpg" | "cmpl"
    \alt <condop>
\end{grammar}
\label{fig:grmJimpleExpr}
\caption{Jimple grammar---expressions}
\end{figure}