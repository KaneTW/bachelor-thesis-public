
\documentclass{beamer}

\usetheme{Berlin}
\useoutertheme{infolines}
\usepackage{natbib,bibentry}
\usepackage[utf8]{inputenc}
\usepackage[english]{babel}
\usepackage[T1]{fontenc}
\usepackage{float}
\usepackage{amsmath}
\usepackage{graphicx}
\usepackage{amssymb}
\usepackage{amsthm}
\usepackage{caption}
\usepackage{minted}
\usepackage{parcolumns}
\usepackage[normalem]{ulem}
\usepackage{tikz}
\usepackage{bcprules}
\usepackage{mathrsfs}
\usepackage[beamer,customcolors]{hf-tikz}
\newcommand{\Dom}{\operatorname{Dom}}
\newcommand{\Matching}{\operatorname{access}}

\usetikzlibrary{shapes,snakes, fit}
\usetikzlibrary{scopes,backgrounds}
\usetikzlibrary{calc, positioning, matrix,arrows}

\newminted{java}{linenos}
\newmint{java}{linenos}
\newmintinline{java}{}

\title{Refining heap-shape information for Java programs using reachable types}
\author{David Kr\"{a}utmann}
\institute{RWTH Aachen}


\AtBeginSection[]
  {
     \begin{frame}<beamer>
     \frametitle{Outline}
     \tableofcontents[currentsection]
     \end{frame}
  }



\tikzset{
c2/.style={
set fill color=green!50!lime!60,
set border color=green!40!black,
},
c3/.style={
set fill color=blue!20!cyan!60,
set border color=blue!60!cyan,
}
}


\makeatletter
\def\normaljustify{%
  \let\\\@centercr
  \rightskip\z@skip
  \leftskip\z@skip%
  \parfillskip=0pt plus 1.0fil\relax
}
\makeatother

%TODO either less text or keep presentation relevant to text

\begin{document}

\begin{frame} \titlepage \end{frame}

\begin{frame}{Outline} \tableofcontents \end{frame}

\section{Introduction}
\begin{frame}{Introduction}
\begin{itemize}
    \item Shape analysis is a useful tool in many other analyses
    \begin{itemize}
        \item Termination checking
        \item Compiler optimisation
        \item Automatic parallelisation
    \end{itemize}
    \item Goal: improve shape analysis accuracy using type information
    \item Existing reachable type analyses are not modular and path-sensitive
    %TODO too general
    
\end{itemize}
\end{frame}
\begin{frame}{Reachable type analysis}
\begin{itemize}
    \item Goal: model what types are reachable from a given reference
    \item Associate each type with a pattern describing where it can occur
    \item[$\Rightarrow$] Path-sensitive
    \item Ensure that analysis has to run only once for each method
    \item[$\Rightarrow$] Modular
    % \item Path-sensitive, sound, interprocedual type analysis    
    % \item Each local variable/static field is assigned a \emph{parametric type set}
    % \item Effect of method invocations on memory state is modeled by \emph{parametric method results}
    % \item Parametric: allows for placeholders for arguments and static fields that can be replaced at invocation sites
    % \item[$\Rightarrow$] Each method has to be analysed only once
\end{itemize}
\end{frame}

\section{Background}
\subsection{Heap paths}
\begin{frame}{Problem}
\begin{itemize}
    \item Heap locations can be represented by a sequence of field and array accesses
    \item The actual path taken to access an object $y$ from $x$ is finite
    \item Problem 1: multiple, potentially infinite paths from $x$ to $y$ exist
    \item Problem 2: static analysis has to converge for all inputs
    \item[$\Rightarrow$] Simple, finite and suitably permissive representation required
\end{itemize}
\end{frame}
\begin{frame}{Definitions}
\begin{itemize}
    \item \emph{heap edge}: instance field \[ \mathcal{H} := \mathcal{F} \]
    \item \emph{heap sequence}: finite word over the alphabet of all heap edges \[ \mathcal{H}^* \]
    \item \emph{heap path}: overapproximating pattern over heap edges \[ \mathcal{P} \]
\end{itemize}
\end{frame}
\begin{frame}{Heap path}
\begin{block}{1. bounded edges}
Heap edge $h^{[n,m]}$ with a lower bound $n \in \mathbb{N}$ and an upper bound $m \in \{ 0,1,* \}$
\begin{itemize}
    \item<1-> $\mathtt{value}^{[0,1]}$
        \begin{itemize}
            \item[$\checkmark$] matches $\epsilon$
            \item[$\checkmark$] matches $\mathtt{value}$
            \item[$\times$] does not match anything else
        \end{itemize}
    \item<2-> $\mathtt{next}^{[1,*]}$
        \begin{itemize}
            \item[$\checkmark$] matches $\mathtt{next}$
            \item[$\checkmark$] matches $\mathtt{next.next}$
            \item[$\times$] does not match $\epsilon$
            \item[$\times$] does not match $\mathtt{value.next}$
        \end{itemize}
\end{itemize}
\end{block}
\end{frame}
\begin{frame}{Heap path}
\begin{block}{2. set of bounded edges}
Set of bounded heap edges: each element can occur in any order
\begin{itemize}
    \item<1-> $\{ \mathtt{left}^{[0,*]}, \mathtt{right}^{[0,*]} \}$ 
        \begin{itemize}
            \item[$\checkmark$] matches $\texttt{left}$
            \item[$\checkmark$] matches $\texttt{right}$
            \item[$\checkmark$] matches $\texttt{left.right.left}$
        \end{itemize}
    \item<2-> $\{ \mathtt{value}^{[1,1]}, \mathtt{next}^{[0,*]} \}$
        \begin{itemize}
        \item[$\checkmark$] matches $\mathtt{value}$
            \item[$\checkmark$] matches $\mathtt{value.next}$
            \item[$\checkmark$] matches $\mathtt{next.value.next}$
            \item[$\times$] does not match $\epsilon$
            \item[$\times$] does not match $\mathtt{next}$
            \item[$\times$] does not match $\mathtt{value.value}$
        \end{itemize}

\end{itemize}
\end{block}
\end{frame}
\begin{frame}{Heap path}
\begin{block}{3. sequence of sets}
Sequence of sets: heap edges occurring in a set cannot occur after it
\begin{itemize}
    \item<1-> $\{\mathtt{next}^{[0,*]}\}, \{ \mathtt{value}^{[1,1]} \}$
        \begin{itemize}
            \item[$\checkmark$] matches $\mathtt{value}$
            \item[$\checkmark$] matches $\mathtt{next.value}$
            \item[$\checkmark$] matches $\mathtt{next.next.value}$
            \item[$\times$] does not match $\epsilon$
            \item[$\times$] does not match $\mathtt{value.next}$
            \item[$\times$] does not match $\mathtt{next}$
            \item[$\times$] does not match $\mathtt{next.value.next}$
        \end{itemize}
\end{itemize}
\end{block}
\end{frame}
\begin{frame}{Heap path}
\begin{block}{4. concrete prefix}
Heap sequence as prefix that always occurs
\begin{itemize}
    \item $\mathtt{left.right} \{ \mathtt{next}^{[1,*]}, \mathtt{value}^{[1,1]} \}$
    \begin{itemize}
    \item[$\checkmark$] matches $\mathtt{left.right.value}$
    \item[$\checkmark$] matches $\mathtt{left.right.value.next}$
    \item[$\checkmark$] matches $\mathtt{left.right.next.next.value}$
    \item[$\times$] does not match $\epsilon$
    \item[$\times$] does not match $\mathtt{left.next}$
    \item[$\times$] does not match $\mathtt{left.right}$
    \item[$\times$] does not match $\mathtt{left.right.next}$
    \end{itemize}
\end{itemize}
\end{block}
\end{frame}
\begin{frame}{Operations}
\begin{itemize}
    \item Alternation and concatenation are not a syntactic element (cf. regular expressions)
    \item Operations that construct a new heap path so that
    \infrule[P-concat]{s_1 \Downarrow p_1 \andalso s_2 \Downarrow p_2}{s_1.s_2 \Downarrow p_1.p_2}
    \infrule[P-alter-left]{s \Downarrow p_1}{s \Downarrow p_1|p_2}
    \infrule[P-alter-right]{s \Downarrow p_2}{s \Downarrow p_1|p_2}
\end{itemize}
\end{frame}
\subsection{Heap-shape analysis}
\begin{frame}{Heap-shape analysis}
\begin{itemize}
    \item The reachable type analysis requires a way to enumerate all local variables that are the predecessors of a reference
    %TODO local variables only
    %TODO: global: ignore static fields and arrays completely
    \item[$\Rightarrow$] Heap-shape analysis required
    \begin{enumerate}
        \item Sound: soundness of reachable type analysis has to be maintained
        \item Path-sensitive: path information for predecessors required for increased precision
    \end{enumerate}
\end{itemize}
\end{frame}
%TODO unneeded here; maybe i
% \subsection{Soot and Jimple}
% \begin{frame}{Soot}
% \begin{itemize}
%     \item Soot: Java framework for program transformation and analysis
%     \item Operates primarily on Java bytecode
%     \item Provides a simple intermediate representation \emph{Jimple}
% \end{itemize}
% \end{frame}
% \begin{frame}{Jimple}
% \begin{itemize}
%     \item Typed 3-address representation of JBC
%     \item 11 statements, 19 instructions total (cf. JBC: with $>200$ instructions)
%     \item Important statements and expressions:
%     \begin{itemize}
%         \item \emph{assignStmt} (local = \emph{rvalue}, local.field = local) and \emph{identityStmt}~(local~=~\texttt{@this})
%         \item \emph{invokeStmt} and \emph{invokeExpr} (\texttt{staticinvoke}, \texttt{virtualinvoke}, ...)
%         \item \emph{ifStmt} (\texttt{if} \emph{conditionExpr} \texttt{goto} label)  and \emph{gotoStmt} (\texttt{goto} label)
%         \item \emph{returnStmt}
%     \end{itemize}
% \end{itemize}
% \end{frame}
\section{Reachable type analysis}
\subsection{Semantics}
\begin{frame}{Heap shape}
    \tikzset{block/.style={shape=rectangle, draw, node distance=-1pt, minimum width = 10em, line width=1pt}}
    \tikzset{headblock/.style={shape=rectangle, draw, node distance=-1pt, minimum width = 10em, line width=3pt}}
    \begin{tikzpicture}
        \node[headblock] (list1) {List};
        \node[left=1em of list1] {};
        \node[block,below=of list1] (next1) {next};
        \node[block,below=of next1] (value1) {value};
        \node[headblock,right=4em of list1] (list2) {List};
        \node[block,below=of list2] (next2) {next};
        \node[block,below=of next2] (value2) {value};
        
        \node[shape=circle, draw, minimum width=2em, line width=1pt, above=1em of list1] (ref) {xs};
        
        \node[headblock, below=2em of value1] (obj1) {SomeClass};
        \node[headblock, below=2em of value2] (obj2) {OtherClass};
        \draw[->] (next1.east) to (list2.west);
        \draw[->] (ref.south) to (list1.north);
        \draw[->] (value1.south) to (obj1.north);
        \draw[->] (value2.south) to (obj2.north);
        \draw[->, bend right, out=210, looseness=1.5] (next2.east) to (list1.north);
    \end{tikzpicture}
\end{frame}
\begin{frame}{Semantics}
\begin{block}{Exact reachable type set}
    The exact reachable type set $\mathcal{RTS}_\text{exact} = \mathcal{JT} \mapsto \mathscr{P}(\mathcal{H}^*)$ is a partial function of Java types to a set of heap sequences at which each type occurs.
\end{block}
\begin{block}{Runtime reachable types}
    The runtime reachable types $\mathcal{RT}_c : \mathit{Refs} \to \mathcal{RTS}_\text{exact}$ of a reference $r$ at a memory state $c$ are defined by following heap edges and collecting all type information
\end{block}
\end{frame}
\begin{frame}{Heap shape II}
    \tikzset{block/.style={shape=rectangle, draw, node distance=-1pt, minimum width = 10em, line width=1pt}}
    \tikzset{headblock/.style={shape=rectangle, draw, node distance=-1pt, minimum width = 10em, line width=3pt}}
    \begin{tikzpicture}
        \node[headblock] (list1) {List};
        \node[left=1em of list1] {};
        \node[block,below=of list1] (next1) {next};
        \node[block,below=of next1] (value1) {value};
        \node[headblock,right=4em of list1] (list2) {List};
        \node[block,below=of list2] (next2) {next};
        \node[block,below=of next2] (value2) {value};
        
        \node[shape=circle, draw, minimum width=2em, line width=1pt, above=1em of list1] (ref) {xs};
        
        \node[headblock, below=2em of value1] (obj1) {SomeClass};
        \node[headblock, below=2em of value2] (obj2) {OtherClass};
        \draw[->] (next1.east) to (list2.west);
        \draw[->] (ref.south) to (list1.north);
        \draw[->] (value1.south) to (obj1.north);
        \draw[->] (value2.south) to (obj2.north);
        \draw[->, bend right, out=210, looseness=1.5] (next2.east) to (list1.north);
    \end{tikzpicture}
\begin{align*}
    \mathcal{RT}_c(\mathtt{xs}) &= \{ \mathtt{List} \to \{ \epsilon, \mathtt{next}, \mathtt{next.next}, \mathtt{next.next.next}, \cdots\}\\
    &\ \ \ ,\ \mathtt{SomeClass} \to \{ \mathtt{value} \}, \mathtt{OtherClass} \to \{ \mathtt{next.value} \} \}
\end{align*}

\end{frame}
\subsection{Flow analysis}
\begin{frame}{Static analysis considerations}
\begin{itemize}
    \item Heap paths to approximate heap sequences
    \item Heap-shape analysis to track fields of an object
    \item Modular results so that each method only has to be analysed once
    \item Interprocedual flow analysis to follow control flow across methods
\end{itemize}
\end{frame}
\begin{frame}{Reachable type set}
\begin{block}{Reachable type set} 
Reachable type set $\mathcal{RTS} = \mathcal{JT} \mapsto \mathcal{P}$: partial function from Java types to \emph{heap paths}
\end{block}
\pause
\begin{itemize}
\item Problem: method invocation context affects reachable type sets
\pause
\item Solution: allow for parameter placeholders
\end{itemize}
\pause
\begin{block}{Parametric type set}
A \emph{parametric type set} $\mathcal{PTS} = (\mathcal{RTS}, \mathbb{N} \mapsto \mathcal{P})$ is a 2-tuple with
\begin{itemize} 
\item a reachable type set $\mathcal{RTS}$ 
\item a partial function $\mathbb{N} \mapsto \mathcal{P}$ that models the occurence of an argument at some heap path
\end{itemize}
\end{block}
\end{frame}
\begin{frame}{Parametric type set example}
\begin{block}{Example: parametric type set application}
\[ pts = (\ \{\ \tikzmarkin<3->{ts}\mathtt{List} \to \{\mathtt{next}^{[0,*]}\}\tikzmarkend{ts}\ \}\ ,\ \{\ \tikzmarkin<4->[c2]{a0}0 \to \epsilon\tikzmarkend{a0}\ ,\  \tikzmarkin<5->[c3]{a2}2 \to \mathtt{value}\tikzmarkend{a2}\ \}\ )\]
\pause
\[ args = \left[\ \tikzmarkin<4->[c2]{pa0} (\varnothing, \varnothing) \tikzmarkend{pa0}\ ,\ (\varnothing, \varnothing)\ ,\ \tikzmarkin<5->[c3]{pa2} (\{ \texttt{Object} \to \epsilon \}, \varnothing)\tikzmarkend{pa2}\ \right]\]
\pause
\[ \operatorname{apply}(pts, args) = (\ \{\ \tikzmarkin<3->{nts}\mathtt{List} \to \{\mathtt{next}^{[0,*]}\}\tikzmarkend{nts}\ ,\ \tikzmarkin<5->[c3]{na2}\mathtt{Object} \to \mathtt{value}\tikzmarkend{na2}\ \}\ ,\ \varnothing\ ) \]
\end{block}
\end{frame}

\begin{frame}{Parametric method result}
\begin{block}{Parametric method result}
A \emph{parametric method result} $\mathcal{PMR} = (\mathcal{PTS}, \mathcal{PTS}^*)$ is a 2-tuple consisting of 
\begin{itemize}
    \item a parametric type set containing the return value's reachable types
    \item a sequence of parametric type sets with the updated arguments' reachable types
\end{itemize}
The operation $\operatorname{apply} : (\mathcal{PMR}, \mathcal{PTS}^*) \to \mathcal{PMR}$ performs $\operatorname{apply}$ on each parametric type set contained within.
\end{block}
\end{frame}
\begin{frame}[fragile]{Analysis example}
\begin{block}{Example}
Consider the following Java code:
\begin{javacode}
public static List setValue(Object value, List xs) {
    xs.value = value;
    return xs;
}
\end{javacode}
Then the parametric method result for \text{List setValue(Object,List)} is \[(\ \underbrace{(\varnothing, \{ 0 \to \mathtt{value}, 1 \to \epsilon\})}_\text{return value type set} , [\ \underbrace{(\varnothing, \{0 \to \epsilon\})}_\text{first argument}, \underbrace{(\varnothing, \{ 0 \to \mathtt{value}, 1 \to \epsilon\})}_\text{second argument}\ ]\ )\]
\end{block}
\end{frame}

\begin{frame}[fragile]{Possible types of a right-hand value}
\begin{block}{Example: local variable}
\begin{javacode}
List xs = new List();
xs.value = new Unit();
Object obj = this;
\end{javacode}
\vspace{-4pt}
$\texttt{Object result = } \tikzmarkin<1->{ptlocal}\mathtt{xs}\tikzmarkend{ptlocal}$

\vspace{5pt}
State after line 3 with $b := a$: 
\begin{align*} a := \{\mathtt{xs} \to (\{\mathtt{List} \to \epsilon,\ \mathtt{Unit} \to \mathtt{value} \}, \varnothing), \mathtt{obj} \to (\varnothing, \{ 0 \to \epsilon \}) \}
\end{align*}

\vspace{5pt} \hline \vspace{5pt}
Calculate the possible types for \texttt{xs} at line 4:
    \[ \operatorname{pt}(a,b,\tikzmarkin<1->{ptlocal2}\mathtt{xs}\tikzmarkend{ptlocal2}) = (\{ \mathtt{List} \to \epsilon,\ \mathtt{Unit} \to \mathtt{value} \}, \varnothing)\]
\end{block}
\end{frame}

\begin{frame}{Possible types of a right-hand value}
\[  \operatorname{pt} : (\mathit{Refs} \to \mathcal{PTS})^2 \times \mathit{rvalue} \to \mathcal{PTS} \]
\begin{itemize}
\item Two partial functions $a, b : \mathit{Refs} \mapsto \mathcal{PTS}$ where $a$ is the analysis state before the current statement and $b$ is the new state
\item A \emph{rvalue} expression --- any expression that can occur to the right of an assignment
\end{itemize}
\begin{block}{Possible types of \emph{rvalue}}
\begin{itemize}
\item \emph{rvalue} is a reference: return parametric type set of reference in $a$
% \item \emph{rvalue} is a regular value: return type set with type of value at $\epsilon$
% \item \emph{rvalue} is a method invoke: method result of invoked method required
\end{itemize}
\end{block}
\end{frame}

\begin{frame}[fragile]{Possible types of a right-hand value}
\begin{block}{Example: instance field}
\begin{javacode}
List xs = new List();
xs.value = new Unit();
Object obj = this;
\end{javacode}
\vspace{-4pt}
$\texttt{Object result = } \tikzmarkin<1->{ptfield}\mathtt{xs.value}\tikzmarkend{ptfield}$

\vspace{5pt}
State after line 3 with $b := a$: 
\begin{align*} a := \{\mathtt{xs} \to (\{\mathtt{List} \to \epsilon,\ \mathtt{Unit} \to \mathtt{value} \}, \varnothing), \mathtt{obj} \to (\varnothing, \{ 0 \to \epsilon \}) \}
\end{align*}

\vspace{5pt} \hline \vspace{5pt}
Calculate the possible types for \texttt{xs} at line 4:
    \[ \operatorname{pt}(a,b,\tikzmarkin<1->{ptfield2}\mathtt{xs.value}\tikzmarkend{ptfield2}) = (\{ \mathtt{Unit} \to \epsilon \}, \varnothing)\]
\end{block}
\end{frame}

\begin{frame}[fragile]{Possible types of a right-hand value}
\begin{block}{Example: regular value}
\begin{javacode}
List xs = new List();
xs.value = new Unit();
Object obj = this;
\end{javacode}
\vspace{-4pt}
$\texttt{Object result = } \tikzmarkin<1->{ptnew}\texttt{new Object}\tikzmarkend{ptnew}$

\vspace{5pt}
State after line 3 with $b := a$: 
\begin{align*} a := \{\mathtt{xs} \to (\{\mathtt{List} \to \epsilon,\ \mathtt{Unit} \to \mathtt{value} \}, \varnothing), \mathtt{obj} \to (\varnothing, \{ 0 \to \epsilon \}) \}
\end{align*}

\vspace{5pt} \hline \vspace{5pt}
Calculate the possible types for \texttt{xs} at line 4:
    \[ \operatorname{pt}(a,b,\tikzmarkin<1->{ptnew2}\texttt{new Object}\tikzmarkend{ptnew2}) = (\{ \mathtt{Object} \to \epsilon \}, \varnothing)\]
\end{block}
\end{frame}

\begin{frame}{Possible types of a right-hand value}
\[  \operatorname{pt} : (\mathit{Refs} \to \mathcal{PTS})^2 \times \mathit{rvalue} \to \mathcal{PTS} \]
\begin{itemize}
\item Two partial functions $a, b : \mathit{Refs} \mapsto \mathcal{PTS}$ where $a$ is the analysis state before the current statement and $b$ is the new state
\item A \emph{rvalue} expression --- any expression that can occur to the right of an assignment
\end{itemize}
\begin{block}{Possible types of \emph{rvalue}}
\begin{itemize}
\item \emph{rvalue} is a reference: return parametric type set of reference in $a$
\item \emph{rvalue} is a regular value: return type set with type of value at $\epsilon$
% \item \emph{rvalue} is a method invoke: method result of invoked method required
\end{itemize}
\end{block}
\end{frame}

\begin{frame}{Possible-types of a right-hand value}
\begin{block}{Example: method invocation}
Parametric method result for \text{List setValue(Object,List)}: \[(\ \underbrace{(\varnothing, \{ \tikzmarkin<2->[c2]{ptiobj4}0\tikzmarkend{ptiobj4} \to \mathtt{value}, \tikzmarkin<3>[c3]{ptixs1}1\tikzmarkend{ptixs1} \to \epsilon\})}_\text{return value type set} , [\ \underbrace{(\varnothing, \{\tikzmarkin<2->[c2]{ptiobj5}0\tikzmarkend{ptiobj5} \to \epsilon\})}_\text{first argument}, \underbrace{(\varnothing, \{ \tikzmarkin<2->[c2]{ptiobj6}0\tikzmarkend{ptiobj6} \to \mathtt{value}, \tikzmarkin<3>[c3]{ptixs2}1\tikzmarkend{ptixs2} \to \epsilon\})}_\text{second argument}\ ]\ )\]
\[ a := \{\tikzmarkin<3>[c3]{ptixs3}\mathtt{xs} \to (\{\mathtt{List} \to \epsilon,\ \mathtt{Unit} \to \mathtt{value} \}, \varnothing)\tikzmarkend{ptixs3}, \tikzmarkin<2->[c2]{ptiobj1}\mathtt{obj} \to (\varnothing, \{ 0 \to \epsilon \})\tikzmarkend{ptiobj1} \}\] \hline
\[ \operatorname{pt}(a,b,\tikzmarkin<1>{ptinvoke}\mathtt{setValue}(\tikzmarkin<2->[c2]{ptiobj7}\mathtt{obj}\tikzmarkend{ptiobj7}, \tikzmarkin<3>[c3]{ptixs4}\mathtt{xs}\tikzmarkend{ptixs4})\tikzmarkend{ptinvoke} ) = \tikzmarkin<3>[c3]{ptiobjf} (\{ \mathtt{List} \to \epsilon,\ \mathtt{Unit} \to \mathtt{value} \}, \{ \tikzmarkin<2->[c2]{ptiobj2}0 \to \mathtt{value}\tikzmarkend{ptiobj2}) \}) \tikzmarkend{ptiobjf} \]
\begin{align*}
    \text{with } b &= \{ \tikzmarkin<3>[c3]{ptixs5}\mathtt{xs} \to (\{ \mathtt{List} \to \epsilon,\ \mathtt{Unit} \to \mathtt{value} \}, \{  \tikzmarkin<2->[c2]{ptiobj8}0 \to \mathtt{value}\tikzmarkend{ptiobj8} \})\tikzmarkend{ptixs5}\\
    &\ \ \  ,\  \tikzmarkin<2->[c2]{ptiobj3}\mathtt{obj} \to (\varnothing, \{ 0 \to \epsilon \})\tikzmarkend{ptiobj3} \} 
\end{align*}
\end{block}
\end{frame}

\begin{frame}{Possible-types of a right-hand value}
\begin{block}{Possible types of a method invoke expression}
\begin{itemize}
\item \emph{rvalue} is a method invoke: analysis of invoked method required
\item[$\Rightarrow$]<2-> Dynamic dispatch --- $rvalue = r.m(arg_1, ...,arg_n)$: all possible concrete implementations have to be analysed
\begin{itemize}
\item $a(r)$ does not contain an argument placeholder: analyse all subclasses of $r$ contained in $a(r)$ where the path matches $\epsilon$
\item $a(r)$ contains a placeholder: analyse all possible subclasses of $r$
\end{itemize}
\item<3-> Result: parametric method result $pmr_{res}$
\item<4-> Instantiate it: $(retPts, argsPts) = \operatorname{apply}(pmr_{res}, (a(arg_i))_{i \le n})$
\item<5-> Update $arg_i$ and predecessors in $b$ with type set in $argsPts$
\item<5-> Return $retPts$
\end{itemize}
\end{block}
\end{frame}

\begin{frame}{Data flow analysis}
\begin{block}{Forward data flow analysis}
\begin{itemize}
\item Traverses the control-flow graph starting at the method entry point
\item Propagate the analysis state by using $b(stmt) := \operatorname{flowThrough}(a(stmt), stmt)$ where $a(stmt)$ is the state before $stmt$ and $b$ is the state after
\item Fixpoint iteration if the graph contains loops, terminate if $b(stmt)$ does not change
\end{itemize}
\end{block}
\end{frame}

\begin{frame}{Interlude: Jimple}
\begin{itemize}
    \item Stackless representation of Java bytecode
    \item 11 statements, 19 instructions total (cf. JBC: with $>200$ instructions)
    \item Important statements and expressions:
    \begin{itemize}
        \item \emph{assignStmt} (local = \emph{rvalue}, local.field = local) and \emph{identityStmt}~(local~=~\texttt{@this})
        \item \emph{invokeStmt} and \emph{invokeExpr}
        \item \emph{ifStmt} (\texttt{if} \emph{conditionExpr} \texttt{goto} label)  and \emph{gotoStmt} (\texttt{goto} label)
        \item \emph{returnStmt}
    \end{itemize}
\end{itemize}
\end{frame}

\begin{frame}{Flow-through function}
\[ \operatorname{flowThrough} : (\mathit{Refs} \mapsto \mathcal{PTS}, \mathit{Stmt}) \to \mathit{Refs} \mapsto \mathcal{PTS} \]
\begin{itemize}
\item Previous state $a : \mathit{Refs} \mapsto \mathcal{PTS}$ and new state $b := a$
\item Jimple statement $stmt$
\end{itemize}
\begin{block}{Flow-through function}
\begin{itemize}
\item \emph{assignStmt} (x = y): 
\begin{itemize}
\item If x is a local, set $b(x) = pt(a,b,y)$
\item Otherwise set $b(r) = b(r) + preds(r).pt(a,b,y)$ for each predecessors $r$
\end{itemize}
\item \emph{identityStmt} (local = \texttt{@this}): Set the corresponding local to a placeholder
\item \emph{invokeStmt}: treat it like in $\operatorname{pt}(a,b,invokeExpr)$
\end{itemize}
\end{block}
\end{frame}

\begin{frame}[fragile]{Data flow analysis example}
\begin{javacode}
List xs = new List();
while(cond) {
    xs.value = "loop";
    xs.next = new List();
    xs = xs.next;
}
// ...
\end{javacode}
\end{frame}

\begin{frame}[fragile]{Data flow analysis example}
\begin{minted}{text}
xs = new List;

loop:
if !cond goto end;
xs.value = "loop";
xs.next = new List;
xs = xs.next;
goto loop;

end:
// ...
\end{minted}
\end{frame}

\begin{frame}{Data flow analysis example}
\tikzset{block/.style={shape=rectangle, draw, node distance=-1pt, minimum width = 30em, line width=1pt}}
\begin{tikzpicture}[scale=0.7, every node/.style={transform shape}]


 \newcommand*\nodeonecolor{white}
 \newcommand*\nodetwocolor{white}
 \newcommand*\nodethreecolor{white}
 \newcommand*\nodefourcolor{white}
 \newcommand*\nodefivecolor{white}
 
 \newcommand*\mycolor{blue!20!cyan!60}
 
 \only<2>{\renewcommand*\nodeonecolor{\mycolor}}
 \only<3,7,11>{\renewcommand*\nodetwocolor{\mycolor}}
 \only<4,8,12>{\renewcommand*\nodethreecolor{\mycolor}}
 \only<5,9>{\renewcommand*\nodefourcolor{\mycolor}}
 \only<6,10>{\renewcommand*\nodefivecolor{\mycolor}}
 

\node[block, fill=\nodeonecolor] (s0) {\texttt{xs = new List;}};
\node[block, below=of s0, fill=\nodeonecolor] (c0) {{\alt<2->{$ \{ xs \to \{ \mathtt{List} \to \epsilon \} \}$}{}}};

\node[block,below=0.5em of c0, fill=\nodetwocolor] (s1) {\texttt{if !cond goto end;}};
\node[block,below=of s1, fill=\nodetwocolor] (c1) { 
\alt<3-6>{$\{ xs \to \{ \mathtt{List} \to \epsilon \} \}$}{\alt<7-10>{$\{ xs \to \{ \mathtt{List} \to \{ \mathtt{next}^{[0,*]} \}, \mathtt{String} \to \mathtt{value} \} \}$}{
\alt<11->{$\{ xs \to \{ \mathtt{List} \to \{ \mathtt{next}^{[0,*]} \}, \mathtt{String} \to \{ \mathtt{next}^{[0,*]} \}, \{ \mathtt{value}^{[1,1]} \} \} \}$}{}}}
};

\node[block,below=0.5em of c1, fill=\nodethreecolor] (s2) {\texttt{xs.value = "loop";}};
\node[block,below=of s2, fill=\nodethreecolor] (c2) {\alt<4-7>{$\{ xs \to \{ \mathtt{List} \to \epsilon, \mathtt{String} \to \mathtt{value} \} \}$}{\alt<8->{$\{ xs \to \{ \mathtt{List} \to \{ \mathtt{next}^{[0,*]} \}, \mathtt{String} \to \{ \mathtt{next}^{[0,*]} \}, \{ \mathtt{value}^{[1,1]} \} \ \}$}{}}
};

\node[block,below=0.5em of c2, fill=\nodefourcolor] (s3) {\texttt{xs.next = new List;}};
\node[block,below=of s3, fill=\nodefourcolor] (c3) {\alt<5-8>{$\{ xs \to \{ \mathtt{List} \to \{ \mathtt{next}^{[0,1]} \}, \mathtt{String} \to \mathtt{value} \} \}$}{\alt<9->{$\{ xs \to \{ \mathtt{List} \to \{ \mathtt{next}^{[0,*]} \}, \mathtt{String} \to \{ \mathtt{next}^{[0,*]} \}, \{ \mathtt{value}^{[1,1]} \} \} \}$}{}}
};
\node[block,below=0.5em of c3, fill=\nodefivecolor] (s4) {\texttt{xs = xs.next;}};
\node[block,below=of s4, fill=\nodefivecolor] (c4) {\alt<6-9>{$\{ xs \to \{ \mathtt{List} \to \{ \mathtt{next}^{[0,*]} \}, \mathtt{String} \to \mathtt{value} \} \}$}{\alt<10->{$\{ xs \to \{ \mathtt{List} \to \{ \mathtt{next}^{[0,*]} \}, \mathtt{String} \to \{ \mathtt{next}^{[0,*]} \}, \{ \mathtt{value}^{[1,1]} \} \} \}$}{}}
};

\node[block, below=2em of c4, fill=\nodethreecolor] (s2b) {\texttt{// ...}};
\node[block,below=of s2b, fill=\nodethreecolor] (c2b) { 
\alt<4-7>{$\{ xs \to \{ \mathtt{List} \to \epsilon \} \}$}{\alt<8-11>{$\{ xs \to \{ \mathtt{List} \to \{ \mathtt{next}^{[0,*]} \}, \mathtt{String} \to \mathtt{value} \} \}$}{
\alt<11->{$\{ xs \to \{ \mathtt{List} \to \{ \mathtt{next}^{[0,*]} \}, \mathtt{String} \to \{ \mathtt{next}^{[0,*]} \}, \{ \mathtt{value}^{[1,1]} \} \} \}$}{}}}
};
\node<12> (phantom) {};

\draw[->] (c0.south) to (s1.north);
\draw[->] (c1.south) to (s2.north);

\draw[->] (c2.south) to (s3.north);
\draw[->] (c3.south) to (s4.north);

\draw[->, bend left] (s1.east) to node[right] {\texttt{goto end;}} (s2b.east);
\draw[->, bend left] (s4.west) to node[left] {\texttt{goto loop;}} (s1.west);
\end{tikzpicture}
\end{frame}

\subsection{Applications}
\begin{frame}{Applications}
\begin{itemize}
\item Call devirtualisation
\begin{itemize}
\item Possible targets for an invoke $b.m()$: types for $b$ at $\epsilon$
\item Only if the type set is fully concrete at $\epsilon$
\end{itemize}
\item Disproving path existence
\begin{itemize}
\item Type of $y$ not contained in RTS for $x$
\item[$\Rightarrow$] $y$ not reachable from $x$
\end{itemize}
\end{itemize}
\end{frame}
\section{Evaluation}
\begin{frame}[allowframebreaks]{Evaluation}
\begin{itemize}
\item Reachable type analysis uses dynamic dispatch resolution
\item Comparison of RTA with and without refinement
\item 4 different test cases\footnote{\url{https://github.com/KaneTW/bachelor-thesis-testcases}}
%todo mention that TPDB/heap-shape analysis should've been used (maybe after results)
\begin{itemize}
\item ResultListTest: invoked method base is a local variable
\item Level1: base is a field of a local variable
\item FiniteListTest: iterates over list constructed without loops and uses elements as base
\item LoopListTest: as FiniteListTest, but list constructed via loop
\end{itemize}
\end{itemize}
\begin{table}[ht]
    \centering
    \begin{tabular}{l||c|c|c|c|c}
     & \multicolumn{5}{c}{Change in }\\
         Test case & $\mathit{count}$ & $\mathit{total}_\mathit{ts}$ &  $\mathit{total}_\mathit{p}$ & $\mathit{avg}_{\mathit{ts}}$ & $\mathit{avg}_{\mathit{p}}$ \\
         \hline
         ResultListTest & 0\% & -23\% & 0\% & -23\% & 0\% \\
         Level1 & -3.4\% & -29\% & 0\% & -26\% & +4.0\% \\
         FiniteListTest & -3.6\% & -9.8\% & 0\% & -6.4\% & +3.8\% \\
         LoopListTest & -2.0\% & -3.8\% & 0\% & -1.8\% & +2.1\% 
    \end{tabular}
    \caption{Result changes with refinement enabled}
    \label{tab:changes}
\end{table}
\end{frame}
\begin{frame}{Evaluation}
\begin{block}{Evaluation was only preliminary}
\begin{itemize}
    \item Originally planned: Integrate with heap-shape analysis and measure refinements
    \item[$\Rightarrow$] Implementation difficulties
    \item Why artificial examples instead of TPDB?
    \item[$\Rightarrow$] TPDB either caused HSA to crash or didn't provide improvements
\end{itemize}
\end{block}
\end{frame}
\section*{Conclusion}
\begin{frame}{Future work}
    \begin{itemize}
        \item Arrays and static fields support
        \item Recursion support
        \item Integration with heap-shape analysis
        \item More exact parametric type sets: include type set only if some method actually occurs in the concrete type set \emph{after parameter replacement}
    \end{itemize}
\end{frame}
\begin{frame}{Conclusion}
\begin{itemize}
    \item Implemented a sound, path-sensitive, modular reachable type analysis
    \item Uses existing heap path formulation and heap-shape analysis
    \item Primary application: refining heap-shape information
    \item Reasonable improvements in call devirtualisation accuracy
\end{itemize}
\end{frame}
\end{document}
 
